\documentclass[../relazione.tex]{subfiles}

\begin{document}
\subsection{Utilizzo I/O su file}
\label{ssec:i-o-file}
Le funzionalità di input/output su file sono state implementate nel caricamento e nel salvataggio di un personaggio.
Sarà dunque possibile durante la partita interropere il gioco e salvare il proprio personaggio su un file JSon contenente i seguenti campi:
\begin{itemize}
    \item \textbf{Caratteristiche}: contiene le caratteristiche del personaggio, ovvero i suoi punti vita, punti mana e il nome;
    \item \textbf{Arma}: contiene le caratteristiche dell'arma. Possono variare a seconda del tipo di arma posseduta(arco,spada o arma magica);
    \item \textbf{Armatura}: contiene le caratteristiche dell'armatura;
    \item \textbf{Inventario}: è una lista di \textbf{Armi} e \textbf{Armature} formate come citato nei punti precendenti.
\end{itemize}

Il programma al momento della creazione di una nuova partita permetterà di caricare un file \textit{.fpg} (formato utilizzato anche per il salvataggio).
Una volta selezionato il file il programma eseguirà i dovuti controlli per verificare che sia ben formato e con contenuti coerenti.
Se la verifica andrà a buon fine avvierà una nuova partita con il personaggio caricato, altrimenti mostrerà una finestra che avvisa dell'errore durante il caricamento.

L'input su file viene utilizzato pure per salvare il punteggio ottenuto dal giocatore alla fine di ogni partita.
Verrá salvato su un file .txt con il seguente formato:\\ 
\textbf{Personaggio: nome   Punteggio: punti}
\end{document}