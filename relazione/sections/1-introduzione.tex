\documentclass[../relazione.tex]{subfiles}

\begin{document}

\subsection{Scopo del progetto}
\label{ssec:scopo-progetto}
Il progetto si prefigge lo scopo di creare un gioco sotto forma di applicazione sviluppata in C++ utilizzando il framework Qt per implementare il front-end.\\ \\
Il gioco, il cui nome è \textit{Dungeons\&Programming2} d'ora in poi \textbf{D\&B2} segue regole dei più classici e semplici giochi di ruolo fantasy, si potrà quindi
creare un personaggio, il quale avrá a sua disposizione un massimo di un'arma e un'armatura per volta, ed un inventario in cui conservare gli oggetti trovati durante
l'avventura con capienza virtualmente infinita.\\
Il nostro personaggio potrá girare liberamente per la mappa, affrontare i nemici (guadagnando così punti che verranno poi salvati nella classifica), raccogliere oggetti
tra cui armature, tre diversi tipi di arma e pozioni che potranno rigenerare la vita o il mana (i punti magia a sua disposizione) del nostro eroe.\\
Inoltre il giocatore potrá anche decidere di salvare il proprio personaggio (ancora in vita) se desidera continuare la partita piú tardi.\\
L'avventura finisce quando il personaggio muore oppure quando il giocatore decide di smettere di giocare, alla fine di ogni partita al giocatore verrá chiesto se desidera salvare il proprio punteggio sulla classifica.
\end{document}